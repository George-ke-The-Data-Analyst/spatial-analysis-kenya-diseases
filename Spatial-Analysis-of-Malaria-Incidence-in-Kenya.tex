% Options for packages loaded elsewhere
\PassOptionsToPackage{unicode}{hyperref}
\PassOptionsToPackage{hyphens}{url}
%
\documentclass[
]{article}
\usepackage{amsmath,amssymb}
\usepackage{iftex}
\ifPDFTeX
  \usepackage[T1]{fontenc}
  \usepackage[utf8]{inputenc}
  \usepackage{textcomp} % provide euro and other symbols
\else % if luatex or xetex
  \usepackage{unicode-math} % this also loads fontspec
  \defaultfontfeatures{Scale=MatchLowercase}
  \defaultfontfeatures[\rmfamily]{Ligatures=TeX,Scale=1}
\fi
\usepackage{lmodern}
\ifPDFTeX\else
  % xetex/luatex font selection
\fi
% Use upquote if available, for straight quotes in verbatim environments
\IfFileExists{upquote.sty}{\usepackage{upquote}}{}
\IfFileExists{microtype.sty}{% use microtype if available
  \usepackage[]{microtype}
  \UseMicrotypeSet[protrusion]{basicmath} % disable protrusion for tt fonts
}{}
\makeatletter
\@ifundefined{KOMAClassName}{% if non-KOMA class
  \IfFileExists{parskip.sty}{%
    \usepackage{parskip}
  }{% else
    \setlength{\parindent}{0pt}
    \setlength{\parskip}{6pt plus 2pt minus 1pt}}
}{% if KOMA class
  \KOMAoptions{parskip=half}}
\makeatother
\usepackage{xcolor}
\usepackage[margin=1in]{geometry}
\usepackage{color}
\usepackage{fancyvrb}
\newcommand{\VerbBar}{|}
\newcommand{\VERB}{\Verb[commandchars=\\\{\}]}
\DefineVerbatimEnvironment{Highlighting}{Verbatim}{commandchars=\\\{\}}
% Add ',fontsize=\small' for more characters per line
\usepackage{framed}
\definecolor{shadecolor}{RGB}{248,248,248}
\newenvironment{Shaded}{\begin{snugshade}}{\end{snugshade}}
\newcommand{\AlertTok}[1]{\textcolor[rgb]{0.94,0.16,0.16}{#1}}
\newcommand{\AnnotationTok}[1]{\textcolor[rgb]{0.56,0.35,0.01}{\textbf{\textit{#1}}}}
\newcommand{\AttributeTok}[1]{\textcolor[rgb]{0.13,0.29,0.53}{#1}}
\newcommand{\BaseNTok}[1]{\textcolor[rgb]{0.00,0.00,0.81}{#1}}
\newcommand{\BuiltInTok}[1]{#1}
\newcommand{\CharTok}[1]{\textcolor[rgb]{0.31,0.60,0.02}{#1}}
\newcommand{\CommentTok}[1]{\textcolor[rgb]{0.56,0.35,0.01}{\textit{#1}}}
\newcommand{\CommentVarTok}[1]{\textcolor[rgb]{0.56,0.35,0.01}{\textbf{\textit{#1}}}}
\newcommand{\ConstantTok}[1]{\textcolor[rgb]{0.56,0.35,0.01}{#1}}
\newcommand{\ControlFlowTok}[1]{\textcolor[rgb]{0.13,0.29,0.53}{\textbf{#1}}}
\newcommand{\DataTypeTok}[1]{\textcolor[rgb]{0.13,0.29,0.53}{#1}}
\newcommand{\DecValTok}[1]{\textcolor[rgb]{0.00,0.00,0.81}{#1}}
\newcommand{\DocumentationTok}[1]{\textcolor[rgb]{0.56,0.35,0.01}{\textbf{\textit{#1}}}}
\newcommand{\ErrorTok}[1]{\textcolor[rgb]{0.64,0.00,0.00}{\textbf{#1}}}
\newcommand{\ExtensionTok}[1]{#1}
\newcommand{\FloatTok}[1]{\textcolor[rgb]{0.00,0.00,0.81}{#1}}
\newcommand{\FunctionTok}[1]{\textcolor[rgb]{0.13,0.29,0.53}{\textbf{#1}}}
\newcommand{\ImportTok}[1]{#1}
\newcommand{\InformationTok}[1]{\textcolor[rgb]{0.56,0.35,0.01}{\textbf{\textit{#1}}}}
\newcommand{\KeywordTok}[1]{\textcolor[rgb]{0.13,0.29,0.53}{\textbf{#1}}}
\newcommand{\NormalTok}[1]{#1}
\newcommand{\OperatorTok}[1]{\textcolor[rgb]{0.81,0.36,0.00}{\textbf{#1}}}
\newcommand{\OtherTok}[1]{\textcolor[rgb]{0.56,0.35,0.01}{#1}}
\newcommand{\PreprocessorTok}[1]{\textcolor[rgb]{0.56,0.35,0.01}{\textit{#1}}}
\newcommand{\RegionMarkerTok}[1]{#1}
\newcommand{\SpecialCharTok}[1]{\textcolor[rgb]{0.81,0.36,0.00}{\textbf{#1}}}
\newcommand{\SpecialStringTok}[1]{\textcolor[rgb]{0.31,0.60,0.02}{#1}}
\newcommand{\StringTok}[1]{\textcolor[rgb]{0.31,0.60,0.02}{#1}}
\newcommand{\VariableTok}[1]{\textcolor[rgb]{0.00,0.00,0.00}{#1}}
\newcommand{\VerbatimStringTok}[1]{\textcolor[rgb]{0.31,0.60,0.02}{#1}}
\newcommand{\WarningTok}[1]{\textcolor[rgb]{0.56,0.35,0.01}{\textbf{\textit{#1}}}}
\usepackage{graphicx}
\makeatletter
\def\maxwidth{\ifdim\Gin@nat@width>\linewidth\linewidth\else\Gin@nat@width\fi}
\def\maxheight{\ifdim\Gin@nat@height>\textheight\textheight\else\Gin@nat@height\fi}
\makeatother
% Scale images if necessary, so that they will not overflow the page
% margins by default, and it is still possible to overwrite the defaults
% using explicit options in \includegraphics[width, height, ...]{}
\setkeys{Gin}{width=\maxwidth,height=\maxheight,keepaspectratio}
% Set default figure placement to htbp
\makeatletter
\def\fps@figure{htbp}
\makeatother
\setlength{\emergencystretch}{3em} % prevent overfull lines
\providecommand{\tightlist}{%
  \setlength{\itemsep}{0pt}\setlength{\parskip}{0pt}}
\setcounter{secnumdepth}{-\maxdimen} % remove section numbering
\ifLuaTeX
  \usepackage{selnolig}  % disable illegal ligatures
\fi
\usepackage{bookmark}
\IfFileExists{xurl.sty}{\usepackage{xurl}}{} % add URL line breaks if available
\urlstyle{same}
\hypersetup{
  pdftitle={Spatial Analysis of major communicable and chronic diseasesin Kenya},
  pdfauthor={CHEGE GEORGE MWANGI (SCT213-C002-0004/2020)},
  hidelinks,
  pdfcreator={LaTeX via pandoc}}

\title{Spatial Analysis of major communicable and chronic diseasesin
Kenya}
\author{CHEGE GEORGE MWANGI (SCT213-C002-0004/2020)}
\date{2025-04-20}

\begin{document}
\maketitle

\textbf{INTRODUCTION}

Understanding the geographic distribution of major communicable and
chronic diseases is critical for targeting public health interventions
in Kenya. In this analysis, we combine county‑level administrative
boundaries with case records for Malaria, Tuberculosis (TB), HIV and
Asthma to:

\begin{enumerate}
\def\labelenumi{\arabic{enumi}.}
\tightlist
\item
  \textbf{Quantify incidence} by computing the number of reported cases
  per disease in each of Kenya's 47 counties.\\
\item
  \textbf{Visualize spatial patterns} via choropleth maps, using color
  ramps tailored to each disease.\\
\item
  \textbf{Compare relative burden} across diseases and regions,
  highlighting hotspots where case counts are especially high.
\end{enumerate}

Our data sources are:

\begin{itemize}
\tightlist
\item
  \textbf{KEN\_adm1 shapefile} from the Kenya National Bureau of
  Statistics, providing up‑to‑date county boundaries.\\
\item
  \textbf{Disease case CSV} (WHO prediction dataset), containing
  individual case records with county identifiers and disease labels.
\end{itemize}

By the end of this report, we will have four clear, standalone
maps---one per disease---and insights into where resources and
interventions might be most urgently needed.

\begin{center}\rule{0.5\linewidth}{0.5pt}\end{center}

\begin{Shaded}
\begin{Highlighting}[]
\NormalTok{packages }\OtherTok{\textless{}{-}} \FunctionTok{c}\NormalTok{(}\StringTok{"sf"}\NormalTok{, }\StringTok{"spData"}\NormalTok{, }\StringTok{"ggplot2"}\NormalTok{, }\StringTok{"dplyr"}\NormalTok{, }\StringTok{"tidyverse"}\NormalTok{)}
\NormalTok{installed }\OtherTok{\textless{}{-}}\NormalTok{ packages }\SpecialCharTok{\%in\%} \FunctionTok{installed.packages}\NormalTok{()}
\FunctionTok{library}\NormalTok{(sf)          }\CommentTok{\# For spatial data handling}
\end{Highlighting}
\end{Shaded}

\begin{verbatim}
## Linking to GEOS 3.13.1, GDAL 3.10.2, PROJ 9.5.1; sf_use_s2() is TRUE
\end{verbatim}

\begin{Shaded}
\begin{Highlighting}[]
\FunctionTok{library}\NormalTok{(tidyverse)   }\CommentTok{\# For data manipulation}
\end{Highlighting}
\end{Shaded}

\begin{verbatim}
## -- Attaching core tidyverse packages ------------------------ tidyverse 2.0.0 --
## v dplyr     1.1.4     v readr     2.1.5
## v forcats   1.0.0     v stringr   1.5.1
## v ggplot2   3.5.2     v tibble    3.2.1
## v lubridate 1.9.4     v tidyr     1.3.1
## v purrr     1.0.4
\end{verbatim}

\begin{verbatim}
## -- Conflicts ------------------------------------------ tidyverse_conflicts() --
## x dplyr::filter() masks stats::filter()
## x dplyr::lag()    masks stats::lag()
## i Use the conflicted package (<http://conflicted.r-lib.org/>) to force all conflicts to become errors
\end{verbatim}

\begin{Shaded}
\begin{Highlighting}[]
\FunctionTok{library}\NormalTok{(ggplot2)     }\CommentTok{\# For visualization}
\FunctionTok{library}\NormalTok{(spdep)       }\CommentTok{\# For spatial autocorrelation}
\end{Highlighting}
\end{Shaded}

\begin{verbatim}
## Loading required package: spData
## To access larger datasets in this package, install the spDataLarge
## package with: `install.packages('spDataLarge',
## repos='https://nowosad.github.io/drat/', type='source')`
\end{verbatim}

\begin{Shaded}
\begin{Highlighting}[]
\FunctionTok{library}\NormalTok{(tmap)        }\CommentTok{\# For thematic mapping}
\FunctionTok{library}\NormalTok{(sp)          }\CommentTok{\# For spatial data classes}
\FunctionTok{library}\NormalTok{(RColorBrewer)}\CommentTok{\# For color palettes}
\FunctionTok{library}\NormalTok{(leaflet)     }\CommentTok{\# For interactive maps}
\end{Highlighting}
\end{Shaded}

\begin{Shaded}
\begin{Highlighting}[]
\CommentTok{\# Load Kenya administrative shape file}
\NormalTok{Counties\_shp }\OtherTok{\textless{}{-}} \FunctionTok{st\_read}\NormalTok{(}\StringTok{"C:/Users/cex/Desktop/data science/DATA SCIENCE 4.2 2025/SDS 2403 R PROGRAMMING FOR BUSINESS ANALYTICS/New folder/KEN\_adm\_shp"}\NormalTok{)}
\end{Highlighting}
\end{Shaded}

\begin{verbatim}
## Multiple layers are present in data source C:\Users\cex\Desktop\data science\DATA SCIENCE 4.2 2025\SDS 2403 R PROGRAMMING FOR BUSINESS ANALYTICS\New folder\KEN_adm_shp, reading layer `KEN_adm0'.
## Use `st_layers' to list all layer names and their type in a data source.
## Set the `layer' argument in `st_read' to read a particular layer.
\end{verbatim}

\begin{verbatim}
## Warning in CPL_read_ogr(dsn, layer, query, as.character(options), quiet, :
## automatically selected the first layer in a data source containing more than
## one.
\end{verbatim}

\begin{verbatim}
## Reading layer `KEN_adm0' from data source 
##   `C:\Users\cex\Desktop\data science\DATA SCIENCE 4.2 2025\SDS 2403 R PROGRAMMING FOR BUSINESS ANALYTICS\New folder\KEN_adm_shp' 
##   using driver `ESRI Shapefile'
## Simple feature collection with 1 feature and 67 fields
## Geometry type: MULTIPOLYGON
## Dimension:     XY
## Bounding box:  xmin: 33.90959 ymin: -4.720417 xmax: 41.92622 ymax: 5.061166
## Geodetic CRS:  WGS 84
\end{verbatim}

\begin{Shaded}
\begin{Highlighting}[]
\CommentTok{\# List all shape file layers}
\FunctionTok{st\_layers}\NormalTok{(}\StringTok{"C:/Users/cex/Desktop/data science/DATA SCIENCE 4.2 2025/SDS 2403 R PROGRAMMING FOR BUSINESS ANALYTICS/New folder/KEN\_adm\_shp"}\NormalTok{)}
\end{Highlighting}
\end{Shaded}

\begin{verbatim}
## Driver: ESRI Shapefile 
## Available layers:
##   layer_name geometry_type features fields              crs_name
## 1   KEN_adm0       Polygon        1     67                WGS 84
## 2   KEN_adm1       Polygon       47     12                WGS 84
## 3   KEN_adm2       Polygon      302     14                WGS 84
## 4   KEN_adm3       Polygon     1446     15                WGS 84
## 5  subcounty       Polygon      302     14 WGS 84 / UTM zone 37S
\end{verbatim}

\begin{Shaded}
\begin{Highlighting}[]
\CommentTok{\# Read counties (KEN\_adm1)}
\NormalTok{counties\_df }\OtherTok{\textless{}{-}} \FunctionTok{st\_read}\NormalTok{(}\StringTok{"C:/Users/cex/Desktop/data science/DATA SCIENCE 4.2 2025/SDS 2403 R PROGRAMMING FOR BUSINESS ANALYTICS/New folder/KEN\_adm\_shp"}\NormalTok{, }
                          \AttributeTok{layer =} \StringTok{"KEN\_adm1"}\NormalTok{)}
\end{Highlighting}
\end{Shaded}

\begin{verbatim}
## Reading layer `KEN_adm1' from data source 
##   `C:\Users\cex\Desktop\data science\DATA SCIENCE 4.2 2025\SDS 2403 R PROGRAMMING FOR BUSINESS ANALYTICS\New folder\KEN_adm_shp' 
##   using driver `ESRI Shapefile'
## Simple feature collection with 47 features and 12 fields
## Geometry type: MULTIPOLYGON
## Dimension:     XY
## Bounding box:  xmin: 33.90959 ymin: -4.720417 xmax: 41.92622 ymax: 5.061166
## Geodetic CRS:  WGS 84
\end{verbatim}

\begin{Shaded}
\begin{Highlighting}[]
\CommentTok{\# Read the healthcare CSV file}
\NormalTok{Diseases\_df }\OtherTok{\textless{}{-}} \FunctionTok{read\_csv}\NormalTok{(}\StringTok{"Kenya Malaria\_Prediction Dataset by who(1).csv"}\NormalTok{)}
\end{Highlighting}
\end{Shaded}

\begin{verbatim}
## Rows: 5000 Columns: 16
## -- Column specification --------------------------------------------------------
## Delimiter: ","
## chr (11): Gender, Region, Fever, Headache, Chills, Sweats, Fatigue, Parasite...
## dbl  (5): Patient ID, Age, Hemoglobin (g/dL), Platelet (cells/?L), WBC (cell...
## 
## i Use `spec()` to retrieve the full column specification for this data.
## i Specify the column types or set `show_col_types = FALSE` to quiet this message.
\end{verbatim}

\begin{Shaded}
\begin{Highlighting}[]
\CommentTok{\# View the structure to understand the data}
\FunctionTok{str}\NormalTok{(Diseases\_df)}
\end{Highlighting}
\end{Shaded}

\begin{verbatim}
## spc_tbl_ [5,000 x 16] (S3: spec_tbl_df/tbl_df/tbl/data.frame)
##  $ Patient ID         : num [1:5000] 1 2 3 4 5 6 7 8 9 10 ...
##  $ Age                : num [1:5000] 52 93 15 72 61 21 83 87 75 75 ...
##  $ Gender             : chr [1:5000] "Male" "Male" "Male" "Female" ...
##  $ Region             : chr [1:5000] "Rural" "Rural" "Rural" "Urban" ...
##  $ Fever              : chr [1:5000] "Yes" "Yes" "Yes" "Yes" ...
##  $ Headache           : chr [1:5000] "No" "Yes" "Yes" "No" ...
##  $ Chills             : chr [1:5000] "Yes" "No" "Yes" "Yes" ...
##  $ Sweats             : chr [1:5000] "No" "Yes" "No" "No" ...
##  $ Fatigue            : chr [1:5000] "Yes" "Yes" "No" "No" ...
##  $ Hemoglobin (g/dL)  : num [1:5000] 12.2 11.5 12.5 12.7 11.6 12.7 10.5 11.2 13.7 10.9 ...
##  $ Platelet (cells/?L): num [1:5000] 367281 261529 383009 245800 379350 ...
##  $ WBC (cells/?L)     : num [1:5000] 10578 6367 11904 11793 10677 ...
##  $ Parasite Detected  : chr [1:5000] "No" "Yes" "No" "No" ...
##  $ Diagnosis          : chr [1:5000] "Negative" "Negative" "Negative" "Negative" ...
##  $ NAME_1             : chr [1:5000] "Tharaka-Nithi" "West Pokot" "Kitui" "Migori" ...
##  $ Diseases           : chr [1:5000] "TB" "Malaria" "HIV" "TB" ...
##  - attr(*, "spec")=
##   .. cols(
##   ..   `Patient ID` = col_double(),
##   ..   Age = col_double(),
##   ..   Gender = col_character(),
##   ..   Region = col_character(),
##   ..   Fever = col_character(),
##   ..   Headache = col_character(),
##   ..   Chills = col_character(),
##   ..   Sweats = col_character(),
##   ..   Fatigue = col_character(),
##   ..   `Hemoglobin (g/dL)` = col_double(),
##   ..   `Platelet (cells/?L)` = col_double(),
##   ..   `WBC (cells/?L)` = col_double(),
##   ..   `Parasite Detected` = col_character(),
##   ..   Diagnosis = col_character(),
##   ..   NAME_1 = col_character(),
##   ..   Diseases = col_character()
##   .. )
##  - attr(*, "problems")=<externalptr>
\end{verbatim}

\begin{Shaded}
\begin{Highlighting}[]
\FunctionTok{head}\NormalTok{(Diseases\_df)}
\end{Highlighting}
\end{Shaded}

\begin{verbatim}
## # A tibble: 6 x 16
##   `Patient ID`   Age Gender Region Fever Headache Chills Sweats Fatigue
##          <dbl> <dbl> <chr>  <chr>  <chr> <chr>    <chr>  <chr>  <chr>  
## 1            1    52 Male   Rural  Yes   No       Yes    No     Yes    
## 2            2    93 Male   Rural  Yes   Yes      No     Yes    Yes    
## 3            3    15 Male   Rural  Yes   Yes      Yes    No     No     
## 4            4    72 Female Urban  Yes   No       Yes    No     No     
## 5            5    61 Male   Rural  No    Yes      Yes    No     No     
## 6            6    21 Male   Rural  Yes   Yes      No     No     Yes    
## # i 7 more variables: `Hemoglobin (g/dL)` <dbl>, `Platelet (cells/?L)` <dbl>,
## #   `WBC (cells/?L)` <dbl>, `Parasite Detected` <chr>, Diagnosis <chr>,
## #   NAME_1 <chr>, Diseases <chr>
\end{verbatim}

\begin{Shaded}
\begin{Highlighting}[]
\CommentTok{\# Merge shape file (counties\_df) with Diseases\_df by county name (NAME\_1)}
\NormalTok{merged\_data }\OtherTok{\textless{}{-}}\NormalTok{ counties\_df }\SpecialCharTok{\%\textgreater{}\%}
  \FunctionTok{left\_join}\NormalTok{(Diseases\_df, }\AttributeTok{by =} \StringTok{"NAME\_1"}\NormalTok{)}

\CommentTok{\# View the merged result}
\FunctionTok{head}\NormalTok{(merged\_data)}
\end{Highlighting}
\end{Shaded}

\begin{verbatim}
## Simple feature collection with 6 features and 27 fields
## Geometry type: MULTIPOLYGON
## Dimension:     XY
## Bounding box:  xmin: 35.52292 ymin: -0.198901 xmax: 36.49007 ymax: 1.660731
## Geodetic CRS:  WGS 84
##   ID_0 ISO NAME_0 ID_1  NAME_1 HASC_1 CCN_1 CCA_1 TYPE_1 ENGTYPE_1 NL_NAME_1
## 1  118 KEN  Kenya    1 Baringo  KE.BA    30  <NA> County    County      <NA>
## 2  118 KEN  Kenya    1 Baringo  KE.BA    30  <NA> County    County      <NA>
## 3  118 KEN  Kenya    1 Baringo  KE.BA    30  <NA> County    County      <NA>
## 4  118 KEN  Kenya    1 Baringo  KE.BA    30  <NA> County    County      <NA>
## 5  118 KEN  Kenya    1 Baringo  KE.BA    30  <NA> County    County      <NA>
## 6  118 KEN  Kenya    1 Baringo  KE.BA    30  <NA> County    County      <NA>
##   VARNAME_1 Patient ID Age Gender Region Fever Headache Chills Sweats Fatigue
## 1      <NA>         14  22   Male  Urban   Yes      Yes     No    Yes      No
## 2      <NA>         63  44 Female  Urban    No       No     No     No      No
## 3      <NA>        134  37   Male  Rural   Yes       No    Yes    Yes      No
## 4      <NA>        138  91 Female  Rural   Yes      Yes    Yes    Yes     Yes
## 5      <NA>        186  75   Male  Urban   Yes       No    Yes     No      No
## 6      <NA>        224  92 Female  Rural    No       No     No     No     Yes
##   Hemoglobin (g/dL) Platelet (cells/?L) WBC (cells/?L) Parasite Detected
## 1              11.6              189396          10420               Yes
## 2              13.5              199202          10569                No
## 3              14.3              183049           5578                No
## 4              12.7              343369           7486                No
## 5              12.7              203258           6945               Yes
## 6              13.2              288155           5718                No
##   Diagnosis Diseases                       geometry
## 1  Positive       TB MULTIPOLYGON (((35.67241 1....
## 2  Negative       TB MULTIPOLYGON (((35.67241 1....
## 3  Negative      HIV MULTIPOLYGON (((35.67241 1....
## 4  Positive      HIV MULTIPOLYGON (((35.67241 1....
## 5  Negative       TB MULTIPOLYGON (((35.67241 1....
## 6  Negative  Malaria MULTIPOLYGON (((35.67241 1....
\end{verbatim}

\begin{Shaded}
\begin{Highlighting}[]
\CommentTok{\# Count cases per condition per county}
\NormalTok{diseases\_ken }\OtherTok{\textless{}{-}}\NormalTok{ Diseases\_df }\SpecialCharTok{\%\textgreater{}\%}
  \FunctionTok{group\_by}\NormalTok{(NAME\_1, Diseases) }\SpecialCharTok{\%\textgreater{}\%}
  \FunctionTok{summarise}\NormalTok{(}\AttributeTok{Cases =} \FunctionTok{n}\NormalTok{(), }\AttributeTok{.groups =} \StringTok{"drop"}\NormalTok{)}
\CommentTok{\# List of diseases to map}
\NormalTok{diseases }\OtherTok{\textless{}{-}} \FunctionTok{c}\NormalTok{(}\StringTok{"Malaria"}\NormalTok{, }\StringTok{"TB"}\NormalTok{, }\StringTok{"HIV"}\NormalTok{, }\StringTok{"Asthma"}\NormalTok{)}
\end{Highlighting}
\end{Shaded}

\begin{Shaded}
\begin{Highlighting}[]
\CommentTok{\# Loop through diseases and plot}
\ControlFlowTok{for}\NormalTok{ (disease }\ControlFlowTok{in}\NormalTok{ diseases) \{}
  
  \CommentTok{\# Filter for one disease}
\NormalTok{  Dis\_data }\OtherTok{\textless{}{-}}\NormalTok{ diseases\_ken }\SpecialCharTok{\%\textgreater{}\%}
    \FunctionTok{filter}\NormalTok{(Diseases }\SpecialCharTok{==}\NormalTok{ disease)}
  
  \CommentTok{\# Merge with spatial data}
\NormalTok{  merged\_map }\OtherTok{\textless{}{-}}\NormalTok{ counties\_df }\SpecialCharTok{\%\textgreater{}\%}
    \FunctionTok{left\_join}\NormalTok{( Dis\_data, }\AttributeTok{by =} \StringTok{"NAME\_1"}\NormalTok{)}
\NormalTok{\}}
\end{Highlighting}
\end{Shaded}

\begin{Shaded}
\begin{Highlighting}[]
\CommentTok{\# a named vector of palettes}
\NormalTok{palettes }\OtherTok{\textless{}{-}} \FunctionTok{c}\NormalTok{(}
  \AttributeTok{Malaria =} \StringTok{"YlOrRd"}\NormalTok{,}
  \AttributeTok{TB      =} \StringTok{"Blues"}\NormalTok{,}
  \AttributeTok{HIV     =} \StringTok{"Purples"}\NormalTok{,}
  \AttributeTok{Asthma  =} \StringTok{"Greens"}
\NormalTok{)}

\CommentTok{\# loop: merge + plot in one step}
\ControlFlowTok{for}\NormalTok{ (d }\ControlFlowTok{in} \FunctionTok{names}\NormalTok{(palettes)) \{}
  
  \CommentTok{\# 1) subset your counts  }
\NormalTok{  df\_d }\OtherTok{\textless{}{-}}\NormalTok{ diseases\_ken }\SpecialCharTok{\%\textgreater{}\%} 
    \FunctionTok{filter}\NormalTok{(Diseases }\SpecialCharTok{==}\NormalTok{ d)}
  
  \CommentTok{\# 2) join spatial geometry  }
\NormalTok{  map\_d }\OtherTok{\textless{}{-}}\NormalTok{ counties\_df }\SpecialCharTok{\%\textgreater{}\%} 
    \FunctionTok{left\_join}\NormalTok{(df\_d, }\AttributeTok{by =} \StringTok{"NAME\_1"}\NormalTok{)}
  
  \CommentTok{\# 3) build \& immediately print the plot  }
\NormalTok{  p }\OtherTok{\textless{}{-}} \FunctionTok{ggplot}\NormalTok{(map\_d) }\SpecialCharTok{+}
    \FunctionTok{geom\_sf}\NormalTok{(}\FunctionTok{aes}\NormalTok{(}\AttributeTok{fill =}\NormalTok{ Cases), }\AttributeTok{color =} \StringTok{"white"}\NormalTok{, }\AttributeTok{size =} \FloatTok{0.2}\NormalTok{) }\SpecialCharTok{+}
    \FunctionTok{scale\_fill\_distiller}\NormalTok{(}
      \AttributeTok{palette   =}\NormalTok{ palettes[d],}
      \AttributeTok{direction =} \DecValTok{1}\NormalTok{,}
      \AttributeTok{na.value  =} \StringTok{"grey90"}\NormalTok{,}
      \AttributeTok{name      =} \StringTok{"Count"}
\NormalTok{    ) }\SpecialCharTok{+}
    \FunctionTok{labs}\NormalTok{(}
      \AttributeTok{title    =} \FunctionTok{paste0}\NormalTok{(d, }\StringTok{" Cases by County"}\NormalTok{),}
      \AttributeTok{subtitle =} \StringTok{"Kenya, 2020"}
\NormalTok{    ) }\SpecialCharTok{+}
    \FunctionTok{theme\_minimal}\NormalTok{() }\SpecialCharTok{+}
    \FunctionTok{theme}\NormalTok{(}\AttributeTok{legend.position =} \StringTok{"right"}\NormalTok{)}
  
  \FunctionTok{print}\NormalTok{(p)}
\NormalTok{\}}
\end{Highlighting}
\end{Shaded}

\includegraphics{Spatial-Analysis-of-Malaria-Incidence-in-Kenya_files/figure-latex/unnamed-chunk-8-1.pdf}
\includegraphics{Spatial-Analysis-of-Malaria-Incidence-in-Kenya_files/figure-latex/unnamed-chunk-8-2.pdf}
\includegraphics{Spatial-Analysis-of-Malaria-Incidence-in-Kenya_files/figure-latex/unnamed-chunk-8-3.pdf}
\includegraphics{Spatial-Analysis-of-Malaria-Incidence-in-Kenya_files/figure-latex/unnamed-chunk-8-4.pdf}

\textbf{CONCLUSION}

This spatial analysis reveals distinct geographic patterns for each
disease:

\begin{itemize}
\tightlist
\item
  \textbf{Malaria} shows its highest burdens in the western and coastal
  counties, consistent with known high‑transmission zones.\\
\item
  \textbf{TB} case counts are elevated in urban centers (e.g.~Nairobi,
  Mombasa) and portions of the Rift Valley, likely reflecting both
  population density and health‑system access.\\
\item
  \textbf{HIV} remains concentrated in western counties bordering Lake
  Victoria, aligning with historical prevalence data.\\
\item
  \textbf{Asthma} rates appear more dispersed, with moderate counts in
  both highland and lowland regions---suggesting environmental and
  lifestyle factors at play.
\end{itemize}

These maps can guide policymakers and health practitioners to:

\begin{itemize}
\tightlist
\item
  \textbf{Prioritize resource allocation} (e.g.~bed nets, diagnostic
  clinics) in identified high‑burden areas.\\
\item
  \textbf{Investigate underlying drivers} of regional variation, such as
  climate, socio‑economic status, and health‑service coverage.\\
\item
  \textbf{Plan targeted interventions}, combining spatial analysis with
  demographic and environmental data in future work.
\end{itemize}

\end{document}
